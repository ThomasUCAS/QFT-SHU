\section{引言}

\subsection{引言}

构建OFT的一些初衷

\begin{itemize}
    \item 相对论性量子力学中的负能态问题
    \item 量子力学二级微扰论的虚粒子的诠释
    \item 因果律(causality)
\end{itemize}

\subsubsection{因果律问题的分析}

我们考虑传播子

\begin{equation}
    K(\vec{x}_2,\vec{x}_1; t_2, t_1):=\bra{\vec{x}_2}\hat{U}(t_2, t_1)\ket{\vec{x}_1}
\end{equation}

其中$\hat{U}(t_2, t_1)$是时间演化算符。这一项的物理意义是,假设在t1时刻有一个由$\ket{\vec{x}_1}$来刻画的量子态,即位于$\vec{x}_1$处的坐标算符本征态。让这个态经过一段时间演化后,它会变成弥漫于空间的波,在t2时刻有一定几率处于$\vec{x}_2$,$K(\vec{x}_2,\vec{x}_1; t_2, t_1)$所刻画的正是这个概率。在狭义相对论中,我们知道物质的运动和信息的传播都不能超光速,即类空的两点之间不应该有关联,所以类空的两点之间的传播子应该为0。我们计算非相对论性量子力学中的自由粒子的传播子可以得到:(取Planck常数$\hbar=1$)

\begin{equation}
    \begin{split}
        K(\vec{x}_2,\vec{x}_1; t, 0) &= \bra{\vec{x}_2}e^{-i\frac{\vec{\hat{p}}^2}{2m}t}\ket{\vec{x}_1} \\
            &= \int \frac{d^3p}{(2\pi)^3} \bra{\vec{x}_2}e^{-i\frac{\vec{\hat{p}}^2}{2m}t}\ket{\vec{p}}<\vec{p}|\vec{x}_1> \\
            &= \int \frac{d^3p}{(2\pi)^3} e^{-i\frac{\vec{p}^2}{2m}t} e^{i\vec{p}\dot(\vec{x}_2-\vec{x}_1)} \\
            &= \biggl(\frac{m}{2\pi it}\biggr)^{3/2} e^{im|\vec{x}_2-\vec{x}_1|/2t}
    \end{split}
\end{equation}

当$|\vec{x}_2-\vec{x}_1|$很大而t很小时,这个传播子仍然不为0,说明在类空间隔下的两个点之间存在关联,这与狭义相对论是矛盾的。所以我们应当考虑相对论性量子力学,此时自由粒子的传播子是

\begin{equation}
    \begin{split}
        K(\vec{x}_2,\vec{x}_1; t, 0) &= \bra{\vec{x}_2}e^{-it\sqrt{\vec{p}^2+m^2}}\ket{\vec{x}_1} \\
            &= \int \frac{d^3p}{(2\pi)^3} e^{-it\sqrt{\vec{p}^2+m^2}} e^{i\vec{p}\dot(\vec{x}_2-\vec{x}_1)} \\
            &= \frac{1}{2\pi^2|\vec{x}_2-\vec{x}_1|} \int p\sin(p|\vec{x}_2-\vec{x}_1|)e^{-it\sqrt{\vec{p}^2+m^2}}dp \\
            &\sim e^{-m\sqrt{\vec{x}^2-t^2}}
    \end{split}
\end{equation}

这个传播子在类空间隔下仍然不为0,说明相对论性量子力学在因果律方面存在一些问题,这将在QFT中得到回答。



\subsubsection{构建理论的一般步骤}

\begin{itemize}
    \item[1] 写下拉氏量,比如$\mathcal{L}[\phi]=\partial^\mu\phi\partial_\mu\phi-\frac{1}{2}m^2\phi^2+\lambda\phi^4$
    \item[2] 写下路径积分,比如$Z=\int\mathcal{D}\phi e^{i\int\mathcal{L}[\phi]}$
    \item[3] 对路径积分依照耦合系数进行微扰展开
    \item[4] 微扰计算路径积分
    \item[5] 发现存在发散问题
    \item[6] 正规化来分离发散部分,比如引入截断$\Lambda$,于是$\int_\mathbb{R}\frac{1}{x^2}dx \to \int_{|x|>\frac{1}{\Lambda}\frac{1}{x^2}}dx$
    \item[7] 令耦合系数为截断的微扰展开式
    \item[8] 重整化,只考虑路径积分的有限项
    \item[9] 与实验比较
    \item[10] 拿Nobel奖,或者从头开始   
\end{itemize}



\subsection{分析力学回顾}

\subsubsection{最小作用量原理与Euler-Lagrange方程}

分析力学中,我们用广义坐标$q(t)$与广义速度$\dot{q}=\frac{dq}{dt}$来刻画一个质点系统的运动状态。拉氏量$L(q_i,\dot{q}_i;t)$是关于广义坐标和广义速度的泛函,用来刻画系统的运动规律。

\noindent Note: 函数$f:\mathbb{R}^n\to\mathbb{R}^2$

泛函$L:C(\Omega)\to\mathbb{R}$,其中$C(\Omega)$是$\Omega$上的全体函数

力学体系的作用量定义为

\begin{equation}
    S=\int_{t_1}^{t_2}L(q_i(t),\dot{q}_i(t);t)dt
\end{equation}

最小作用量原理(哈密顿原理):对于真实的一个运动轨迹,当坐标发生一个变分后,作用量不会变小。也就是说$\delta S=0$。

\noindent 注:这里的变分是等时变分,即$\delta t=0$,故拉氏量的变分中没有$\frac{\delta L}{\delta t}\delta t$项。

\begin{equation}
    \begin{split}
        0=\delta S &= \delta\int_{t_1}^{t_2} L(q_i,\dot{q}_i;t)dt \\
            &= \int_{t_1}^{t_2}dt \biggl(\frac{\delta L}{\delta q_i}\delta q_i+\frac{\delta L}{\delta \dot{q}_i}\delta\dot{q}_i\biggr) \\
            &= \int_{t_1}^{t_2}dt \frac{\delta L}{\delta q_i}\delta q_i + \int_{t_1}^{t_2}(\delta\dot{q}_idt)\frac{\delta L}{\delta \dot{q}_i}\delta\dot{q}_i \\
            &= \int_{t_1}^{t_2}dt \frac{\delta L}{\delta q_i}\delta q_i + \frac{\delta L}{\delta \dot{q}_i}\delta q_i|_{t_1}^{t_2} - \int_{t_1}^{t_2}dt \frac{d}{dt}\bigl(\frac{\delta L}{\delta \dot{q}_i}\bigr)\delta q_i \\
            &= \int_{t_1}^{t_2}dt \biggl(\frac{\delta L}{\delta q_i}-\frac{d}{dt}\bigl(\frac{\delta L}{\delta \dot{q}_i}\bigr)\biggr)\delta q_i
    \end{split}
\end{equation}

于是我们得到了Euler-Lagrange方程如下

\begin{equation}\label{Euler-Lagrange}
    \frac{\delta L}{\delta q_i}-\frac{d}{dt}\bigl(\frac{\delta L}{\delta \dot{q}_i}\bigr) = 0
\end{equation}

Euler-Lagrange方程是系统的运动方程,比如对于一个谐振子,其拉氏量为$L=\frac{1}{2}m\dot{x}^2-\frac{1}{2}kx^2$,根据Euler-Lagrange方程可以得到,这正是谐振子的运动方程$m\ddot{x}+kx=0$。

考虑Newton第二定律$\vec{F}=m\ddot{\vec{x}}$,对于保守力$\vec{F}$,一般总可以写成势能的梯度$\vec{F}=\nabla V$,于是Newton第二定律可以写成$m\ddot{\vec{x}}-\nabla V$。这与Euler-Lagrange方程具有相同的形式,即如果令$L=\frac{1}{2}m\dot{x}^2-V$,则Euler-Lagrange方程给出的正是Newton第二定律,并且我们顺便得到了在保守力系统中,$L=T-V$。

\subsubsection{哈密顿力学}

定义广义动量

\begin{equation}\label{Legendre}
    p_i=\frac{\partial L}{\partial \dot{q}_i} 
\end{equation}

对拉氏量作Legendre变换,定义哈密顿量,它是广义坐标和广义动量的函数

\begin{equation}
    H(q_i,p_i;t)=p_i\dot{q}_i-L
\end{equation}

由于L不显含$p_i$,我们计算得到$\frac{\partial H}{\partial p_i}=\dot{q}_i$;以及根据Euler-Lagrange方程(\ref{Euler-Lagrange}),$\frac{\partial H}{\partial q_i}=-\frac{\partial L}{\partial q_i}=-\frac{d}{dt}\frac{\partial L}{\partial \dot{q}_i}=-\frac{d}{dt}q_i=-\dot{q}_i$。于是我们得到了哈密顿正则方程

\begin{equation}\label{Hamilton-eq}
    \left\{
        \begin{array}{lr}
            \dot{p}_i = -\frac{\partial H}{\partial q_i} \\
            \dot{q}_i = \frac{\partial H}{\partial p_i}
        \end{array}
    \right.
\end{equation}

对于保守力系统,我们已经证明了其拉氏量等于动能减势能,即$L=T-V$。由(\ref{Legendre}),不难得到$H=T+V$。应注意此式仅对保守力学系统成立,对于经典电磁场下带电质点,其哈密顿量为$H=\frac{(\vec{p}+e\hat{A})^2}{2m}+e\phi(x)$,就不再是$T+V$的形式了。

对于一个谐振子,不难计算得到其哈密顿量$H=\frac{p^2}{2m}+\frac{1}{2}kx^2$。利用哈密顿正则方程,不难得到谐振子的运动方程是

\begin{equation}
    \left\{
        \begin{array}{lr}
            \dot{p}_i = -kx \\
            \dot{q}_i = \frac{p}{m}
        \end{array}
    \right.
\end{equation}

定义泊松括号

\begin{equation}
    \{A,B\}=\frac{\partial A}{\partial q_i}\frac{\partial B}{\partial p_i}-\frac{\partial B}{\partial q_i}\frac{\partial A}{\partial p_i}
\end{equation}

这里默认对指标i求和,称为Einstein求和。当然,严格来讲Einstein求和是一上一下两个指标求和,称为缩并,本质上来讲是流形的切空间及其对偶空间的内积,这里的$q_i$和$p_i$并不构成流形的切空间和对偶空间,所以这里不强调上下标。当然,更精细的理论会从辛几何的角度来理解分析力学,此时拉格朗日力学是切丛上的力学,哈密顿力学是余切丛上的力学,Einstein求和也确实是切空间和余切空间的内积。于是可以得到广义坐标和广义动量的对易关系

\begin{equation}
    \begin{split}
        \{q_i, p_j\} &= \frac{\partial q_i}{\partial q_k}\frac{\partial p_j}{\partial q_k}-\frac{\partial p_j}{\partial q_k}\frac{\partial q_i}{\partial p_k} \\
            &= \delta_{ik}\delta_{jk} \\
            &= \delta_{ij}
    \end{split}
\end{equation}

力学量$F=F(q_i,p_i;t)$的演化方程为

\begin{equation}
    \begin{split}
        \frac{dF}{dt} &= \frac{\partial F}{\partial q_i}\dot{q}_i+\frac{\partial F}{\partial p_i}\dot{p}_i+\frac{\partial F}{\partial t} \\
            &= \frac{\partial F}{\partial q_i}\frac{\partial H}{\partial p_i}-\frac{\partial H}{\partial q_i}\frac{\partial F}{\partial p_i}+\frac{\partial F}{\partial t} \\
            &= \{F,H\}+\frac{\partial F}{\partial t}
    \end{split}
\end{equation}

特别地,对于哈密顿量,我们有

\begin{equation}
    \frac{d H}{dt}=\frac{\partial H}{\partial t}
\end{equation}

这意味着$\frac{\partial H}{\partial t}=0\Rightarrow\frac{dH}{dt}=0$,也就是说如果哈密顿量不显含时,那么哈密顿量是一个守恒量。



\subsection{量子力学中的谐振子}

一维谐振子的哈密顿量$\hat{H}=\frac{\hat{p}^2}{2m}+\frac{1}{2}m\omega^2\hat{x}^2=-\frac{\hbar^2}{2m}\nabla^2\frac{1}{2}m\omega^2x^2$。定义产生湮灭算符

\begin{equation}
    \left\{
        \begin{array}{lr}
            \hat{a}=\sqrt{\frac{m\omega}{2\hbar}}(\hat{x}+\frac{i}{m\omega}\hat{p})   \\
            \hat{a}^\dagger=\sqrt{\frac{m\omega}{2\hbar}}(\hat{x}-\frac{i}{m\omega}\hat{p})
        \end{array}
    \right.
\end{equation}

产生与湮灭算符的基本性质是

\begin{equation}
    \begin{split}
        \hat{a}\ket{n}=\sqrt{n}\ket{n-1} \\
        \hat{a}^\dagger\ket{n}=\sqrt{n+1}\ket{n+1}
    \end{split}
\end{equation}

于是可以求出第n能级的态矢量为

\begin{equation}
    \ket{n}=\frac{(\hat{a}^\dagger)^n}{\sqrt{n!}}\ket{0}
\end{equation}

定义粒子数算符$\hat{N}=\sum_{i}\hat{a}^\dagger\hat{a}$,于是

\begin{equation}
    \hat{N}\ket{n}=n\ket{n}
\end{equation}

\subsubsection{无耦合多体谐振子与自由全同粒子的粒子数表象}

N个无耦合谐振子的哈密顿量和态空间(Fock态)为

\begin{equation}
    \begin{split}
        \hat{H} = \hat{H}_k = \sum\frac{\hat{p}^2_k}{2m_k}+\frac{1}{2}m_k\omega_k\hat{x}_k^2 = \sum(\hat{a}^\dagger\hat{a}+\frac{1}{2})\hbar\omega_k \\
        \ket{n_1n_2...n_k...n_N}=(\hat{a}_1^\dagger)^{n_1}(\hat{a}_2^\dagger)^{n_2}...(\hat{a}_N^\dagger)^{n_N}
    \end{split}
\end{equation}

多体无耦合谐振子的Fock态空间可以用于刻画自由全同粒子系统。不严格来讲,可以按照以下表格来对应

\begin{center}
    \begin{tabular}{c|c}
        \hline
            谐振子系统 & 全同粒子系统 \\
        \hline
            第k个谐振子 & 第m个粒子 \\
        \hline
            $E=\sum_{k=1}^Nn_k\hbar\omega_k$ & $E=\sum_{m=1}^Nn_mE_m$ \\
        \hline
            Quanta in oscillators & particles in momentum \\
        \hline
        \end{tabular}
\end{center}

真空态定义为$\ket{\Omega}=\ket{00}$,于是$\hat{a}_{p_1}^\dagger\hat{a}_{p_2}^\dagger\ket{\Omega}\propto\ket{11}$,以及$\hat{a}_{p_2}^\dagger\hat{a}_{p_1}^\dagger\ket{\Omega}\propto\ket{11}$。我们可以得到

\begin{equation}
    \hat{a}_{p_1}^\dagger\hat{a}_{p_2}^\dagger=\lambda\hat{a}_{p_2}^\dagger\hat{a}_{p_1}^\dagger,\ |\lambda|=1
\end{equation}

其中$\lambda=1$时为玻色子,$\lambda=-1$时为费米子,$\lambda=e^{i\theta}$为任意子。玻色子和费米子产生湮灭算符的对易关系为

\begin{equation}
    \begin{split}
        [\hat{a}_i^\dagger,\hat{a}_j^\dagger]=[\hat{a}_i,\hat{a}_j]=0 \\
        [\hat{a}_i,\hat{a}_j^\dagger]=\delta_{ij} \\
        \ket{n_1n_2...}=\Pi_m\frac{1}{\sqrt{n_m!}}(\hat{a}_{p_m}^\dagger)^{n_m}\ket{0}
    \end{split}
\end{equation}

\begin{equation}
    \begin{split}
        \{\hat{c}_i^\dagger,\hat{c}_j^\dagger\}=\{\hat{c}_i,\hat{c}_j\}=0 \\
        \{\hat{c}_i,\hat{c}_j^\dagger\}=\delta_{ij}
    \end{split}
\end{equation}


\subsubsection{多体耦合谐振子}

经典耦合谐振子的哈密顿量

\begin{equation}
	H=\sum_{j}\frac{p_j^2}{2m}+\frac{1}{2}k(q_{j+1}-q_j)^2
\end{equation}

这里广义坐标$q_j$是第j个谐振子绝对位置相对于其平衡位置的相对坐标,我们假定相邻两个谐振子平衡位置的距离是a,于是第j个谐振子的平衡位置是ja,它的绝对位置是$x_j=q_j+ja$。对于两个耦合谐振子的情形,我们可以对坐标$(q_1,q_2)$做可逆变换

\begin{equation}
    \left(\begin{matrix}
        q'_1 \\ q'_2
    \end{matrix}
    \right) = \frac{1}{\sqrt{2}}\left(
    \begin{matrix}
        1 & 1 \\
        1 & -1 
    \end{matrix} \right) \left(
    \begin{matrix}
        q_1 \\ q_2
    \end{matrix}    
    \right)
\end{equation}

对动量也做相同变换,我们不难将哈密顿量改写成无耦合的形式$H=\frac{p_1^2}{2m}+\frac{p_2^2}{2m}+\frac{1}{2}m\omega_1^2 q_1^2+\frac{1}{2}m\omega_2^2 q_2^2$。但是如果用相同的方式处理N个耦合谐振子,这个可逆矩阵并不容易构造,于是我们尝试傅里叶变换的方法。从泛函分析的观点来看,傅里叶变换实际上是希尔伯特空间上的基矢变换。我们不难知道,全体满足哈密顿正则方程的$(q_1(t),...,q_n(t);p_1(t),...,p_n(t))$张成一个有限维线性空间,因此是一个希尔伯特空间。我们取这个线性空间的一组基$\{(e^{i(ka-\omega t)},...,e^{i(kja-\omega t)},..., e^{i(kNa-\omega t)};e^{i(ka-\omega t)},...,e^{i(kja-\omega t)},...,e^{i(kNa-\omega t)}):k\in\mathbb{R}\}$。我们对系统要求周期性边界条件$x_{N+j}=x_{j}$,进而对波矢k有一个限制条件:

\begin{equation}
    e^{ikNa}=1
\end{equation}

于是有$k=\frac{2n\pi}{Na}$,共有N种允许的取值,恰好可以描述系统的N个独立自由度。于是对$q_j(t)$和$p_j(t)$可做线性展开,即离散傅里叶变换

\begin{equation}
	\begin{array}{lr}
		q_j=\frac{1}{\sqrt{N}}\sum_{k}\tilde{q}_je^{-ikja} \\
		q_j=\frac{1}{\sqrt{N}}\sum_{k}\tilde{p}_je^{ikja}
	\end{array}
\end{equation}

$e^{i(kja-\omega t)}$称为波矢k的振动模式或第k个振动模式,接下来将看到,它代表了一个集体激发,在固体物理中称为格波,波矢k是格波波矢,对格波做正则量子化后得到声子。于是哈密顿量变成

\begin{equation}
	\hat{H}=\sum_{k}\frac{1}{2m}\tilde{p}_k\tilde{p}_{-k}+\frac{1}{2}m\omega_k^2\tilde{q}_k^2
\end{equation}

这是N个无耦合的谐振子的系统哈密顿量,其中声子的振动频率$\omega_k^2=\frac{4K}{m}\sin^2\frac{ka}{2}$。接下来我们对其进行正则量子化,引入产生湮灭算符

\begin{equation}
	\begin{array}{lr}
		\tilde{x}_k=\sqrt{\frac{\hbar}{2m\omega_k}}(\hat{a}_k+\hat{a}_k^\dagger) \\
		\tilde{p}_k=-i\sqrt{\frac{2m\hbar\omega_k}{2}}(\hat{a}_k-\hat{a}_k^\dagger)
	\end{array}
\end{equation}

于是哈密顿量变成

\begin{equation}
	\hat{H}=\sum_{k}^N\frac{1}{2}\hbar\omega_k(\hat{a}_k\hat{a}_k^\dagger+\hat{a}_{-k}^\dagger\hat{a}_{-k})
\end{equation}

由于$\hat{a}_{-k}^\dagger\hat{a}_{-k}=\hat{a}_{-k}\hat{a}_{-k}^\dagger+1$,于是哈密顿量进而变成

\begin{equation}
	\hat{H}=\sum_{k}^N\hbar\omega_k(\hat{a}_k\hat{a}_k^\dagger+\frac{1}{2})
\end{equation}

其中,当系统的谐振子数N趋于无穷时,第二项会贡献一个无穷大的零点能,不具有可观测效应,这个之后我们再详细讨论。





\subsection{二次量子化:波函数算符化}

不严格地来讲,一次量子化可以理解为把粒子变成波,二次量子化可以理解为把波变成粒子。(二次量子化:对于多个自由粒子,将具有相同状态的粒子平面波构成的波包视为状态特定的“粒子”?产生湮灭算符作用在一个波包上面产生或者湮灭一个平面波。)。

在粒子数表象下,波函数变为算符,作用在多粒子态上在x处产生确定的具有不同动量的粒子。

\begin{equation}
	\left\{
	\begin{array}{lr}
		\Psi^\dagger(x)=\frac{1}{\sqrt{V}}\sum_{p}\hat{a}_p^\dagger e^{-i\vec{p}\cdot\vec{x}} \\
		\Psi(x)=\frac{1}{\sqrt{V}}\sum_{p}\hat{a}_pe^{-i\vec{p}\cdot\vec{x}}
	\end{array}
	\right.
\end{equation}

玻色子和费米子分别满足对易关系和反对易关系

\begin{equation}
	\begin{split}
		[\Psi(\vec{x}),\Psi^\dagger(\vec{y})]=\delta^{(3)}(\vec{x}-\vec{y}) \\
		[\Psi(\vec{x}),\Psi(\vec{y})]=[\Psi^\dagger(\vec{x}),\Psi^\dagger(\vec{y})]=0
	\end{split}
\end{equation}

\begin{equation}
	\begin{array}{lr}
		\{\Psi(\vec{x}),\Psi^\dagger(\vec{y})\}=\delta^{(3)}(\vec{x}-\vec{y}) \\
		\{\Psi(\vec{x}),\Psi(\vec{y})\}=\{\Psi^\dagger(\vec{x}),\Psi^\dagger(\vec{y})\}=0
	\end{array}
\end{equation}

于是算符可以表达为

\begin{equation}
	\begin{split}
		\hat{A} &= \sum_{\alpha\beta}\ket{\alpha}\bra{\alpha}\hat{A}\ket{\beta}\bra{\beta} \\
			&= \sum_{\alpha\beta}A_{\alpha\beta}\ket{\alpha}\bra{\beta} \\
			&= \sum_{\alpha\beta}A_{\alpha\beta}\hat{a}_\alpha^\dagger\hat{a}_\beta
	\end{split}
\end{equation}

特别地,动量算符、哈密顿量算符(自由粒子)、势能算符表达为

\begin{equation}
	\begin{array}{lr}
		\hat{p}= \sum_{p}\vec{p}\ \hat{a}_{p}^\dagger\hat{a}_{p}=\sum_{p}\vec{p}\ \hat{n}_{p}\\
		\hat{H}= \sum_{p}\frac{\vec{p}^2}{2m}\hat{a}_{p}^\dagger\hat{a}_{p}=\sum_{p}\frac{\vec{p}^2}{2m}\hat{n}_{p}\\
		\hat{V}= \sum_{p_{1},p_{2}}V_{\vec{p_{1}}-\vec{p_{2}}}\hat{a}_{p}^\dagger\hat{a}_{p}
           =\sum_{p_{1},p_{2}}V_{\vec{p_{1}}-\vec{p_{2}}}\hat{n}_{p}\\
	\end{array}
\end{equation}




\subsection{多体系统的连续极限与场概念的引入}

多体谐振子的哈密顿量和拉氏量是

\begin{equation}
	\begin{array}{lr}
		H=\sum_{j}(\frac{p_j^2}{2m}+\frac{1}{2}k(q_{j+1}-q_j)^2) \\
		L=\sum_{j}(\frac{p_j^2}{2m}-\frac{1}{2}k(q_{j+1}-q_j)^2)
	\end{array}
\end{equation}

当$N\to\infty$时,我们可以认为$\frac{q_{j+1}-q_j}{l}\to\frac{\partial\phi(t,\vec{x})}{\partial x}\sim\nabla\phi$,即连续极限下,离散的格点可以过渡到连续的场。此时,动能和势能分别过渡到

\begin{equation}
    \begin{split}
        T &= \sum_j\frac{1}{2}m\bigl(\frac{\partial q_i}{\partial t}\bigr)^2 \to \frac{1}{l}\int dx\frac{1}{2}\bigl(\frac{\partial\phi(t,\vec{x})}{\partial t}\bigr)^2 = \int dx\frac{1}{2}\rho\bigl(\frac{\partial\phi}{\partial t}\bigr)^2 \\
        V &= \sum_j\frac{1}{2}k(q_{j+1}-q_i)^2 \to \sum_jkl^2\bigl(\frac{\partial\phi(t,\vec{x})}{\partial \vec{x}}\bigr)^2 \sim \int dx \tau(\nabla\phi)^2
    \end{split}
\end{equation}

这样,哈密顿量和拉氏量过渡到连续体系中变成

\begin{equation}
    \begin{split}
        H &= \int d^3\bigl[\frac{1}{2}\rho\bigl(\frac{\partial\phi}{\partial t}\bigr)^2+\frac{1}{2}\tau(\nabla\phi)^2\bigr] \\
        L &= \int d^x\bigl[\frac{1}{2}\rho\bigl(\frac{\partial\phi}{\partial t}\bigr)^2-\frac{1}{2}\tau(\nabla\phi)^2\bigr]
    \end{split}
\end{equation}

场论中我们希望所有可观测量满足Lorentz协变性,但哈密顿量和拉氏量在Lorentz变换下都不是协变量,这为我们研究可观测量带来不便。我们可以引入拉氏密度$\mathcal{L}=\frac{1}{2}\rho\bigl(\frac{\partial\phi}{\partial t}\bigr)^2-\frac{1}{2}\tau(\nabla\phi)^2$,这是一个Lorentz协变量,于是拉氏量可以表示为$L=\int d^x \mathcal{L}$。这样,作用量就可以写成Lorentz协变的形式

\begin{equation}
    S=\int dtL=\int d^4x\mathcal{L}
\end{equation}

最小作用量原理给出场论中的Euler-Lagrange方程

\begin{equation}
    \frac{\partial\mathcal{L}}{\partial\phi}-\partial_\mu\frac{\partial\mathcal{L}}{\partial(\partial_\mu\phi)}=0
\end{equation}

仿照分析力学,我们可以引入$\phi$场的正则共轭动量场,以及哈密顿量密度

\begin{equation}
    \begin{split}
        \pi(x)=\frac{\partial\mathcal{L}}{\partial\dot{\phi}} \\
        \mathcal{H}=\pi\dot{\phi}-\mathcal{L}
    \end{split}
\end{equation}

类似有正则方程

\begin{equation}
    \left\{
        \begin{array}{lr}
            \dot{\phi}=\frac{\partial\mathcal{H}}{\partial\pi} \\
            \dot{\pi}=-\frac{\partial\mathcal{H}}{\partial\phi}+\nabla\cdot\frac{\partial\mathcal{H}}{\partial(\nabla\phi)}
        \end{array}
    \right.
\end{equation}



\subsection{relativistic}

我们采用记号$\eta^{\mu\nu}=diag(1,-1,-1,-1)$的度规。

张量运算复习

考虑闵氏时空,坐标 $x^{\mu}=(t,\vec{x})$ ,$x_{\mu}=g_{\mu\nu}x^{\nu}=(t,\vec{-x})$。取自然单位制 $c=\hbar=1$ ,此时
\begin{equation}
   \begin{array}{lr}
     p^{\mu}=(E,\vec{p})\\
     p^{2}=p^{\mu}p^{\mu}\eta_{\mu\mu}=E^{2}-\vec{p}^{2}=m^{2}\\
     \partial^{2}=\partial_{\mu}\partial^{\mu}=\partial^{\mu}\partial_{\mu}
   \end{array}
\end{equation}




常见的场

\begin{itemize}
    \item[实标量场] Klein-Gorden方程 $(\partial^2+m^2)\phi(x)=0$

        拉氏量$\mathcal{L}=\frac{1}{2}(\partial^\mu\phi)(\partial_\mu\phi)-\frac{1}{2}m^2\phi^2$

    \item[复标量场] $\phi=\frac{1}{\sqrt{2}}(\phi_1+i\phi_2)$,$\phi^*=\frac{1}{\sqrt{2}}(\phi_1-i\phi_2)$
        
        拉氏量$\mathcal{L}=(\partial^\mu\phi^*)(\partial_\mu\phi)-m^2\phi^*\phi$
        
    \item[Dirac场]
      
    \item[电磁场] $\mathcal{L}=-\frac{1}{4}F_{\mu\nu}F^{\mu\nu}-J_\mu A^\mu$
        
        \begin{equation}
            F^{\mu\nu}=\Biggl(\begin{matrix}
                0 & E_1 & E_2 & E_3 \\
                -E_1 & 0 & -B_3 & B_2 \\
                -E_2 & B_3 & 0 & -B_1 \\
                -E_3 & -B_2 & B_1 & 0 \\
            \end{matrix}\Biggr)
        \end{equation}
        
        因此$F_{\mu\nu}F^{\mu\nu}=2(\vec{B}^2-\vec{E}^2)$
        
        除$F_{\mu\nu}F^{\mu\nu}$外,$F_{\mu\nu}$还可构造出一个Lorentz赝标量$\epsilon^{\mu\nu\alpha\beta}F_{\mu\nu}F_{\alpha\beta}\propto\vec{E}\cdot\vec{B}$,这一项会破坏宇称,与量子反常和拓扑效应有密切的关系。

        U(1)对称性与流守恒
        \begin{equation}
            \partial_\mu J^\mu=0
        \end{equation}

        Maxwell方程组
        \begin{equation}\label{Maxwell1}
            \partial_\lambda F^{\lambda\mu}=J^\mu
            \ \Rightarrow\ \left\{
                \begin{split}
                    \nabla\cdot\vec{E}&=\rho \\
                    \nabla\times\vec{B}&=\vec{J}+\frac{\partial\vec{E}}{\partial t}
                \end{split}
                \right. 
        \end{equation}
        \begin{equation}\label{Maxwell2}
            \partial_\lambda G^{\lambda\mu}=0
            \ \Rightarrow\ \left\{
                \begin{split}
                    \nabla\cdot\vec{B}&=0 \\
                    \nabla\times\vec{E}&=-\frac{\partial\vec{B}}{\partial t}
                \end{split}
            \right.
        \end{equation}

        其中$G_{\mu\nu}=\epsilon_{\mu\nu\alpha\beta}F^{\alpha\beta}$是$F_{\mu\nu}$的对偶(Hodge duality)。此外,(\ref{Maxwell2})也可以由$F^{\mu\nu}$的Bianchi恒等式$$\partial_\lambda F_{\mu\nu}+\partial_\mu F_{\nu\lambda}+\partial_\nu F_{\lambda\mu}=0$$得到。

        我们知道在经典电磁学中,如果引入磁单极子,那么在(\ref{Maxwell2})的右侧需要引入磁荷密度和磁流密度,即
        \begin{equation}
            \partial_\lambda G^{\lambda\mu}=J_m^\mu
            \ \Rightarrow\ \left\{
                \begin{split}
                    \nabla\cdot\vec{B}&=\rho_m \\
                    \nabla\times\vec{E}&=\vec{J}_m-\frac{\partial\vec{B}}{\partial t}
                \end{split}
            \right.
        \end{equation}


\end{itemize}
